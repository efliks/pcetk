\documentclass[a4paper,11pt]{article}

% \usepackage{graphicx, overcite, supertabular, color, colordvi, setspace, amssymb, rotating}
\usepackage{xspace, hyperref}

\pagestyle{plain}
% \pagestyle{empty}

\linespread{1.6}
\textwidth       16 cm
\textheight      22 cm
\oddsidemargin    0 mm 
\evensidemargin   0 mm 
\topmargin      -13 mm

% Prevent indenting paragraphs
\setlength\parindent{0pt}

%=================================================
\newcommand{\modulename}{ContinuumElectrostatics\xspace}



%=================================================
\begin{document}

\begin{center}
{\LARGE \bf \modulename}

\vspace{0.5cm}
{\large User Manual\\
Last update: \today}

\end{center}

\vspace{2cm}


%=================================================
\section{Introduction}
%=================================================
Proteins contain residues, cofactors and ligands that bind or release protons 
depending on the current pH and the interactions with their molecular 
environment.
%
These titratable residues, cofactors and ligands will be referred to as sites.
%
The titration of proteins is often difficult to study experimentally because 
the available methods, such as calorimetry, cannot determine protonation states 
of individual sites.
%
The knowledge of these individual protonation states is crucial for 
understanding of many important processes, for example enzyme catalysis.


The \modulename module extends the functionality of the pDynamo 
library with a Poisson-Boltzmann continuum electrostatic model that allows for 
calculations of protonation states of individual sites.
%
The module provides an interface between pDynamo and the external solver of 
the Poisson-Boltzmann equation, MEAD.
%
MEAD is a program developed by Donald Bashford and extended by Timm Essigke
and Thomas Ullmann.
%
The electrostatic energy terms obtained with MEAD can be used to calculate 
energies of all possible protonation states of the protein of interest.
%
However, analytic evaluation of protonation state energies is only possible 
for proteins with only a few titratable sites. 
%
The \modulename module also provides an interface to the GMCT 
program by Matthias Ullmann and Thomas Ullmann that can be used to sample 
protonation state energies using a Monte Carlo method. 
%
The GMCT interface allows for studying the titration of larger proteins.


%=================================================
% \section{Goals of the \modulename module}
%=================================================
% 
% The module is similar in behaviour to the multiflex2qmpb.pl program by Timm 
% Essigke.
% 
% For the calculated thermodynamic cycle, see Fig. 3.8, p. 82 in Timm's thesis.
% 
% A pqr file is not needed because the atomic radii are assigned to atoms at 
% runtime.


%=================================================
\section{Copying}
%=================================================
The module is distributed under the CeCILL Free Software License, which is
a French equivalent of the GNU General Public License.
%
For details, see the files \texttt{Licence\_CeCILL\_V2-en.txt} (or 
\texttt{Licence\_CeCILL\_V2-fr.txt} for the French version).


%=================================================
\section{Installation and configuration}
%=================================================
Before the installation of the \modulename module, it is necessary 
to have:
\begin{itemize}
  \setlength{\itemsep}{2pt}
  \item pDynamo 1.8.0
  \item Python 2.7
  \item GCC (any version should be fine)
  \item Extended MEAD 2.3.0
  \item GMCT 1.2.3
\end{itemize}
%
Extended MEAD and GMCT can be found on the website of Thomas Ullmann:

\url{http://www.bisb.uni-bayreuth.de/People/ullmannt/index.php?name=software}

Download the two packages and follow their respective installation 
instructions.
%
The \modulename module requires for its functioning two programs 
from the MEAD package, namely \texttt{my\_2diel\_solver} and \texttt{my\_3diel\_solver}, 
and the GMCT's main program, \texttt{gmct}.

\bigskip
In the next step, check out the latest source code of the module.
%
Note that for checking out the source code you should have Subversion installed 
as well.

\texttt{svn checkout \\
http://pdynamo-extensions.googlecode.com/svn/trunk/ContinuumElectrostatics/}


\bigskip
Some parts of the module implementing the state vector are written in C and
therefore have to be compiled before use. In the future, I plan to shift some 
more parts of the code from Python to C.

\bigskip
Go to the subdirectory \texttt{extensions/csource} and edit the Makefile. Only the
uppermost line has to be changed. It defines the directory where you have
installed pDynamo. After editing, close the file and type in \texttt{make} to compile
the C object file.

\bigskip
Go to the subdirectory \texttt{extensions/pyrex} and again edit the Makefile. Change
the line starting from "INC2" to the location of your pDynamo installation.
Close the file and type in \texttt{make}. It should generate a dynamically linked 
library \texttt{StateVector.so} in the \texttt{ContinuumElectrostatics} subdirectory.

\bigskip
At this point, the installation is complete.

\bigskip
Before using the module, you have to set the environment variable\\
\texttt{PDYNAMO\_CONTINUUMELECTROSTATICS} to the directory where you have checked out 
the source code. Also, you have to add this directory to the \texttt{PYTHONPATH}
variable. For example, you can do it this way (in Bash):

\bigskip
\texttt{export \\PDYNAMO\_CONTINUUMELECTROSTATICS=/home/mikolaj/devel/ContinuumElectrostatics}

\texttt{export PYTHONPATH=\$PYTHONPATH:\$PDYNAMO\_CONTINUUMELECTROSTATICS}


%=================================================
\section{Usage}
%=================================================
After the installation, it may be worth looking at some of the test cases in 
the subdirectory \texttt{tests}. I will explain the functioning of the module based 
on the test case "sites2". This test uses a trivial polypeptide with only two
titratable sites, histidine and glutamate. The test "histidine" uses only one
site. The other tests use real-life proteins.

\bigskip
In the first step, prepare CHARMM topology (psf) and coordinate (crd) files. 

\bigskip
During the initial setup in CHARMM, all titratable residues in the protein 
should be set to their standard protonation states at pH = 7, i.e. aspartates 
and glutamates deprotonated, histidines doubly protonated, other residues 
protonated.


%=================================================
\section{References}
%=================================================
MEAD website: \\
\url{http://stjuderesearch.org/site/lab/bashford/}

Extended MEAD website:\\ 
\url{http://www.bisb.uni-bayreuth.de/People/ullmannt/index.php?name=extended-mead}

Doctoral thesis of Timm Essigke:\\
\url{https://epub.uni-bayreuth.de/655/}


%=================================================
\section{Test cases}
%=================================================


%=================================================
\section{To-do list}
%=================================================
\begin{itemize}
  \setlength{\itemsep}{2pt}
  \item Move parts of WriteJobFiles to the instance class
  \item Rename variables containing filenames so that they start from "file"
  \item Coordinates from the FPT file should have their own data structure
  \item Efficiency improvements during writing job files
  \item Use arrays instead of lists for interactions (Real1DArray or SymmetricMatrix)
  \item Have a column of ETA (Estimated Time for Accomplishment) in MEAD calculations
  \item Optionally convert kcal/mol (MEAD units) to kJ/mol (pDynamo units)
  \item Make use of the pqr2SolvAccVol program to speed up the calculations a little bit
  \item The function calculating Gmicro should be written in C
\end{itemize}

\end{document}
