\documentclass[a4paper,11pt]{article}

% \usepackage{graphicx, overcite, supertabular, color, colordvi, setspace, amssymb, rotating}
\usepackage{xspace, hyperref, listings}

\pagestyle{plain}
\linespread{1.6}
\textwidth        16 cm
\textheight       22 cm
\oddsidemargin     0 mm
\evensidemargin    0 mm
\topmargin       -13 mm

% Prevent indenting paragraphs
\setlength\parindent{0pt}

\lstset{basicstyle=\ttfamily, frame=shadowbox, breaklines=true, breakatwhitespace=true}

\newcommand{\modulename}{ContinuumElectrostatics\xspace}


%=================================================
\begin{document}

\begin{center}
{\LARGE \bf \modulename}

\vspace{0.5cm}
{\Large User Manual\\
Last update: \today}

\end{center}

\vspace{2cm}


%=================================================
\section{Introduction}
%=================================================
Proteins contain residues, cofactors and ligands that bind or release protons
depending on the current pH and the interactions with their molecular
environment.
%
These titratable residues, cofactors and ligands will be referred to as sites.
%
The titration of proteins is often difficult to study experimentally because
the available methods, such as calorimetry, cannot determine protonation states
of individual sites.
%
The knowledge of these individual protonation states is crucial for
understanding of many important processes, for example enzyme catalysis.


The \modulename module extends the functionality of the pDynamo
library with a Poisson-Boltzmann continuum electrostatic model that allows for
calculations of protonation states of individual sites.
%
The module provides an interface between pDynamo and the external solver of
the Poisson-Boltzmann equation, MEAD.
%
MEAD is a program developed by Donald Bashford and extended by Timm Essigke
and Thomas Ullmann.
%
The electrostatic energy terms obtained with MEAD can be used to calculate
energies of all possible protonation states of the protein of interest.
%
However, analytic evaluation of protonation state energies is only possible
for proteins with only a few titratable sites.
%
The \modulename module also provides an interface to the GMCT
program by Matthias Ullmann and Thomas Ullmann that can be used to sample
protonation state energies using a Monte Carlo method.
%
The GMCT interface allows for studying the titration of larger proteins.


In the present document, I focus on the practical side of the
electrostatic calculations.
%
Other people have done a great work to develop the theory and computational
methods the \modulename module is based on.
%
%The primary source of information was the doctoral thesis of Timm Essigke.


%=================================================
\section{Copying}
%=================================================
The module is distributed under the CeCILL Free Software License, which is
a French equivalent of the GNU General Public License.
%
For details, see the files \texttt{Licence\_CeCILL\_V2-en.txt} (or
\texttt{Licence\_CeCILL\_V2-fr.txt} for the French version).


%=================================================
\section{Goals of the \modulename module}
%=================================================
I wrote the \modulename module primarily as a pretext to learn how
the Poisson-Boltzmann model works in detail.
%
I used this model very often during my studies on enzyme catalysis but never had
time nor will to learn the details of the theory that was behind.
%
I also wanted to better explore the pDynamo library, understand its programming
concepts and finally extend it with something useful.
%
Last but not least, I saw some room for improvement in the previously used
tools and scripts.

The \modulename module is similar in behaviour to the \texttt{multiflex2qmpb.pl}
program, which is part of the QMPB package written by Timm Essigke.
%
The approach taken here is most compatibile with the original approach
by Donald Bashford.
%
The key difference is that the treatment of multiprotic sites, such as histidine,
is improved.

% Talk about non-binary state vector and different energy reference (fully deprotonated system)

%
% For the calculated thermodynamic cycle, see Fig. 3.8, p. 82 in Timm's thesis.


%=================================================
\section{Installation and configuration}
%=================================================
Before the installation of the \modulename module, it is necessary
to have:
\begin{itemize}
  \setlength{\itemsep}{2pt}
  \item pDynamo 1.8.0
  \item Python 2.7
  \item GCC (any version should be fine)
  \item Extended MEAD 2.3.0
  \item GMCT 1.2.3
\end{itemize}
%
Extended MEAD and GMCT can be found on the website of Thomas Ullmann:

\url{http://www.bisb.uni-bayreuth.de/People/ullmannt/index.php?name=software}

Download the two packages and follow their respective installation
instructions.
%
The \modulename module requires for its functioning two programs
from the MEAD package, namely \texttt{my\_2diel\_solver} and \texttt{my\_3diel\_solver},
and the GMCT's main program, \texttt{gmct}.

\bigskip
In the next step, check out the latest source code of the module.
%
Note that for checking out the source code you should have Subversion installed
as well.

{\footnotesize \begin{lstlisting}
svn checkout http://pdynamo-extensions.googlecode.com/svn/trunk/ContinuumElectrostatics/
\end{lstlisting} }

\bigskip
Some parts of the module implementing the state vector are written in C and
therefore have to be compiled before use. 
%
In the future, I plan to shift some more parts of the module from Python to C.

\bigskip
Start from going to the directory \texttt{extensions/csource}.
%
Change the uppermost line in the \texttt{Makefile}. 
%
This line defines the directory where you have installed pDynamo. 
%
After editing, close the file and run "make" to compile the C object file.

\bigskip
Go to the directory \texttt{extensions/pyrex} and again edit the \texttt{Makefile}.
%
Change the line starting from "INC2" to the location of your pDynamo installation.
%
Close the file and run "make".
%
It should generate a dynamically linked library \texttt{StateVector.so} in 
the \texttt{ContinuumElectrostatics} directory.

\bigskip
At this point, the installation is complete.

\bigskip
Before using the module, the environment variable \texttt{PDYNAMO\_CONTINUUMELECTROSTATICS} 
should be set to the module's root directory.
%
This directory should be also added to the \texttt{PYTHONPATH} variable. 
%
This can be done in the following way (in Bash):

\newpage
{\footnotesize \begin{lstlisting}
export PDYNAMO_CONTINUUMELECTROSTATICS=/home/mikolaj/devel/ContinuumElectrostatics

export PYTHONPATH=$PYTHONPATH:$PDYNAMO_CONTINUUMELECTROSTATICS
\end{lstlisting} }

%=================================================
\section{Usage}
%=================================================
After the installation, it may be worth looking at some of the test cases.
%
I will explain the functioning of the module based on the test case "sites2".
%
This test uses a trivial polypeptide with only two titratable sites,
histidine and glutamate.
%
The test "histidine" uses only one site. The other tests use real-life,
although small proteins.


%-------------------------------------------------
\subsection{Setup of the protein model}
The electrostatic model used by the \modulename module requires that the protein
of interest is described by the CHARMM energy model.
%
In the first step, prepare CHARMM topology (PSF) and coordinate (CRD) files.
%
The preperation can be done using the programs CHARMM or VMD.
%
During the preparation of the protein model, all titratable residues in the protein
should be set to their standard protonation states at pH = 7, i.e. aspartates
and glutamates deprotonated, histidines doubly protonated, other residues
protonated.
%
The topology, coordinate and parameter files are loaded at the
beginning of the script \texttt{sites2.py}:

{\footnotesize \begin{lstlisting}
par_tab = ["charmm/toppar/par_all27_prot_na.inp", ]

mol  = CHARMMPSFFile_ToSystem ("charmm/testpeptide_xplor.psf", isXPLOR = True, parameters = CHARMMParameterFiles_ToParameters (par_tab))

mol.coordinates3 = CHARMMCRDFile_ToCoordinates3 ("charmm/testpeptide.crd")
\end{lstlisting} }


%-------------------------------------------------
\subsection{Setup of the continuum electrostatic model}
In the second step, a continuum electrostatic model is created:

{\footnotesize \begin{lstlisting}
ce_model = MEADModel (meadPath = "/home/mikolaj/local/bin/", gmctPath = "/home/mikolaj/local/bin/", scratch = "scratch", nthreads = 2)
\end{lstlisting} }

\bigskip
The parameter "meadPath" tells the directory where the MEAD programs,
\texttt{my\_2diel\_solver} and \texttt{my\_3diel\_solver}, are located.
%
The parameter "gmctPath" is the location of the \texttt{gmct} program.
%
If none of these directories are given, \texttt{/usr/bin} is assumed
by default.
%
The parameter "scratch" tells the directory where the MEAD job files
and output files will be written to.
%
If not present, this directory will be created.
%
The last parameter, "nthreads", defines the number of threads to be used.
%
By default \texttt{nthreads=1}, which means serial run.
%
Note that "nthreads" can be any natural number and that the calculations
scale linearly with the number of threads.
%
Parallelization is done at the coarse-grain level.
%
Since the electrostatic energy terms for a particular instance of a titratable
site can be calculated independently from energy terms of other instances of other
sites, each instance is assigned a separate thread.
%
A similar approach is taken during the calculations of titration curves,
where each pH-step of a curve is calculated separately.


\bigskip
In the next step, the continuum electrostatic model is initialized:

{\footnotesize \begin{lstlisting}
ce_model.Initialize (mol)
\end{lstlisting} }

\bigskip
The initialization means partitioning of the protein into titratable sites and
a non-titratable background.
%
It also means generating model compounds.
%
At this point, however, the input files for MEAD are not written and only
the necessary data structures inside the \texttt{MEADModel} object are created.
%
The \texttt{Initialize} method takes at least one argument,
which indicates the CHARMM-based protein model.

\bigskip
The next two lines generate a summary of the continuum electrostatic model and
write a table of titratable sites:

{\footnotesize \begin{lstlisting}
ce_model.Summary ()
ce_model.SummarySites ()
\end{lstlisting} }

\bigskip
After the model has been initialized, the input files necessary for calculations in MEAD
can be written to the scratch directory:

{\footnotesize \begin{lstlisting}
ce_model.WriteJobFiles (mol)
\end{lstlisting} }

By default, each site is assigned a separate directory, for example
\texttt{scratch/PRTA/GLU8}.
%
Inside the directory, there are PQR files for each instance of the site in the protein
and in a model compound.
%
The other type of files are MGM and OGM files describing lattices used for
solving the Poisson-Boltzmann equation.
%
The \texttt{back.pqr} file defines the non-titratable background.
%
The \texttt{protein.pqr} file defines the whole protein and is used to calculate the boundary
between the protein and the solvent.
%
The last file, \texttt{sites.fpt}, contains atomic coordinates and charges of all instances
and is used for calculating interaction energies between
different instances of sites in the protein.


%-------------------------------------------------
\subsection{Calculating electrostatic energies}
At this point, the electrostatic energy terms can be calculated:

{\footnotesize \begin{lstlisting}
ce_model.CalculateElectrostaticEnergies ()
\end{lstlisting} }

For each instance of each site, two electrostatic energy terms are calculated, namely
the Born energy ($G_{\mathrm{Born}}$) and the background energy ($G_{\mathrm{back}}$).
%
Born energy is the electrostatic energy of a set of charges interacting with its own
reaction field.
%
Background energy is the electrostatic energy of a set of charges interacting with
charges from outside of this set.
%
The two energies are calculated for a particular instance of a site both in the model compound
and in the protein.
%
The difference
$(G_{\mathrm{Born, protein}} + G_{\mathrm{back, protein}}) - (G_{\mathrm{Born, model}} + G_{\mathrm{back, model}})$
is calculated, which is called the homogeneous transfer energy, $G_{\mathrm{homotrans}}$.
%
Transferring of a site means moving it from the model compound to the protein.
%
In the model compound, the site has a model energy $G_{\mathrm{model}}$, which
corresponds to the experimentally known $\mathrm{p}K_{\mathrm{a}}$ value of the deprotonation reaction.
%
The site in the protein has an intrinsic energy $G_{\mathrm{intr}} = G_{\mathrm{model}} + G_{\mathrm{homotrans}}$.
%
The \texttt{my\_2diel\_solver} program calculates $G_{\mathrm{Born, model}}$ and $G_{\mathrm{back, model}}$.
%
The \texttt{my\_3diel\_solver} program calculates $G_{\mathrm{Born, protein}}$ and $G_{\mathrm{back, protein}}$
and, additionally, electrostatic interaction energies of an instance of a site
with other instances of other sites in the protein.
%
The \modulename module collects 
$G_{\mathrm{Born, protein}}$, $G_{\mathrm{back, protein}}$, 
$G_{\mathrm{Born, model}}$, $G_{\mathrm{back, model}}$ and interaction energies from MEAD 
and calculates $G_{\mathrm{homotrans}}$ and $G_{\mathrm{intr}}$ for each instance of each site.

Note that the \texttt{my\_3diel\_solver} program can in principle perform calculations
in a three-dielectric environment (solvent, protein, vacuum).
%
However, the model implemented in the \modulename module only deals with two-dielectric environments, 
i.e. the solvent phase and the protein/model compound phase.


%-------------------------------------------------
\subsection{Calculating microstate energies}

After the $G_{\mathrm{intr}}$ values and interaction energies have been calculated 
for all instances of all titratable sites, 
one can calculate the energy of a particular protonation state of the protein, i.e. 
the microstate energy, $G_{\mathrm{micro}}$.
%
The polypeptide in the "sites2" example contains only two sites, 
glutamate and histidine,
so there can be $2^1 * 4^1 = 8$ possible protonation states, because glutamate has two
instances ("p" and "d") and histidine has four instances ("HSP", "HSD", "HSE", "fully deprotonated").
%
For real-life proteins the number of possible protonation states is usually very large.



%-------------------------------------------------
\subsection{Calculating protonation probabilities}

The protonation probabilities are calculated at a given pH.
%
This can be done analytically for small proteins or using GMCT for larger proteins.

{\footnotesize \begin{lstlisting}
ce_model.CalculateProbabilitiesAnalytically ()
ce_model.SummaryProbabilities ()
\end{lstlisting} }


%=================================================
\section{References}
%=================================================
MEAD website: \\
\url{http://stjuderesearch.org/site/lab/bashford/}

Extended MEAD website:\\
\url{http://www.bisb.uni-bayreuth.de/People/ullmannt/index.php?name=extended-mead}

Doctoral thesis of Timm Essigke:\\
\url{https://epub.uni-bayreuth.de/655/}


%=================================================
\section{Test cases}
%=================================================


%=================================================
\section{To-do list}
%=================================================
\begin{itemize}
  \setlength{\itemsep}{2pt}
  \item Move parts of WriteJobFiles to the instance class
  \item Rename variables containing filenames so that they start from "file"
  \item Coordinates from the FPT file should have their own data structure
  \item Efficiency improvements during writing job files
  \item Use arrays instead of lists for interactions (Real1DArray or SymmetricMatrix)
  \item Have a column of ETA (Estimated Time for Accomplishment) in MEAD calculations
  \item Optionally convert kcal/mol (MEAD units) to kJ/mol (pDynamo units)
  \item Make use of the pqr2SolvAccVol program to speed up the calculations a little bit
  \item The function calculating Gmicro should be written in C
\end{itemize}

\end{document}
